\documentclass[UTF8]{ctexart}

\usepackage[a4paper,left=2cm,right=2cm,bottom=1cm]{geometry} %版边距
\usepackage{fancyhdr}        %页眉页脚
\usepackage{tcolorbox}		 %彩色文本框
\usepackage{xcolor-material} %调色板
\usepackage{listings}		 %代码提示
\usepackage{hyperref}		 %超链接
\usepackage{titlesec}		 %标题类型样式
\usepackage{titletoc} 		 %目录
\usepackage{siunitx} \usepackage{amssymb}

\colorlet{PrimaryColor}{MaterialBlue900}

\titleformat*{\section}{\bfseries\large\color{PrimaryColor}}
\titleformat*{\subsection}{\bfseries\normalsize\color{PrimaryColor}}
\titleformat*{\subsubsection}{\bfseries\small\color{PrimaryColor}}

\title{Cell WorkStation Document}
\author{Cell Team}
\date{\today}
\begin{document}
	\maketitle
	
	\newpage
	\tableofcontents
		
	\newpage
	\section{概览}
		\subsection{简介}
		Cell WorkStation\texttrademark 是一个完全Developed in-house的深度学习框架,它的结构并不繁复,却十分强大。其底层完全由C++构造,性能优异,上层接入受欢迎的Python语言。同时,Cell WorkStation还集成了一个开发环境,现代、优美而易用。
		\subsection{特性}
			\begin{itemize}
				\item[·] 由Qt framework编写
				\item[·] 现代的界面设计与交互
				\item[·] 内嵌全功能代码编辑器
				\item[·] 明亮和深色模式全局切换
				\item[·] 优异的引擎性能
			\end{itemize}
		\subsection{Cell User Interface}
		Cell WorkStation\texttrademark 包含了一系列自定义的组件基类来实现特殊的功能或视觉效果。这些基类均继承自QWidget。
			\subsubsection{class customDialog}
			为弹出式子对话框实现的高层基类,继承自QDialog。在原有基础上增加了“color”属性来实现全局颜色变换;调用Windows原生窗口阴影。
			\subsubsection{class customStaticButton}
			为静态按钮实现的高层基类,继承自QPushButton。在原有基础上增加了“color”属性来实现全局颜色变换,同时按钮三态能动态变化。
			\subsubsection{class customDynamicButton}
			为动态按钮实现的中层基类,继承自customStaticButton。在定制化静态按钮基础上实现鼠标移入移出的动态变换效果。
			\subsubsection{class customFrame}
			自定义的Frame高层基类,在原有基础上增加了“color”属性来实现全局颜色变换。
			\subsubsection{class customGradientChangeFrame}
			中层基类,继承自customFrame。实现了状态切换的动态效果。
			\subsubsection{class customLabel}
			高层基类,继承自QLabel。在原有基础上增加了“color”属性来实现全局颜色变换。
			\subsubsection{class customWidget}
			高层基类,继承自QWidget。在原有基础上增加了“color”属性来实现全局颜色变换;调用Windows原生窗口阴影。
		\subsection{Cell Calculating Library}
	\section{安装指南}
	访问\url{https://www.riverbankcomputing.com/software/qscintria/downloadpage}下载QScintria for windows。

	解压QScintria源码包后,进入其子目录Qt4Qt5,然后双击项目文件“qscintria.pro”。
	
	Qt将自动载入QScintria源代码,请选择MinGW 32位工具链,分别编译调试版本和发布版本。
	
	之后,你将得到两个动态链接库 \textbf{qscintria2\textunderscore qt5.dll} 和 \textbf{qscintria2\textunderscore qt5d.dll},它们分别对应于调试模式和发布模式。
	
	差不多完成了,将这两个dll放到一个合适的目录中,并将它们添加到cell的pro文件中进行编译。
\end{document}